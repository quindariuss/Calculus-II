\documentclass{article}
\title{Calculus II: Test Three}
\author{Quin'darius Lyles-Woods}
\usepackage{amsmath}
\usepackage[margin=0.5in]{geometry}
\begin{document}
\maketitle
\Large
\section{Integration by Parts}
\begin{equation}
	\int
	x^4 \ln x dx
\end{equation}
\vspace{3.25in}
\begin{equation}
	\int
	x \cos(2x) dx
\end{equation}
\pagebreak
\section{Trigonometric Substitution}
\begin{equation}
	\int 
	\frac{x}
	{\sqrt{9-x^2}}dx
\end{equation}
\vspace{3.5in}
\begin{equation}
	\int 
	\frac{1}
	{\sqrt{x^2-4}}
	dx
\end{equation}
\pagebreak
\section{Divergence Test}
\begin{equation}
	\sum
	\frac
	{n}
	{n+1}
\end{equation}
\pagebreak
\section{Geometric Series}
Determine the form $\sum^\infty_{n=1} ar^{n-1}$ and show whether it is convergent or divergent.
\begin{equation}
	\frac{1}{2}+
	\frac{1}{4}+
	\frac{1}{8}+
	\frac{1}{16}+...
\end{equation}
\vspace{3.5in}
\begin{equation}
	\sum
	6(\tfrac{1}{3})^n
\end{equation}
\pagebreak
\section{P-Series(Hyper Harmonic Series)}
Are the equations convergent or divergent.
\begin{equation}
	\sum
	\frac
	{1}
	{n^2}
\end{equation}
\vspace{2in}
\begin{equation}
	\sum
	\frac
	{1}
	{n}
\end{equation}
\vspace{2in}
\begin{equation}
	1+
	\frac{1}{4}+
	\frac{1}{9}+
	\frac{1}{16}+
	\frac{1}{25}+...
\end{equation}
\vspace{2in}
\begin{equation}
	\sum
	n^{-\frac{1}{2}}
\end{equation}
\pagebreak
\section{Integral Test}
\begin{equation}
	\sum
	\frac{n^2}{n^3+1}
\end{equation}
\vspace{3in}
\begin{equation}
	\sum
	\frac{1}
	{n^2+1}
\end{equation}
\pagebreak
\section{Comparison Test}
\begin{equation}
	\sum
	\frac
	{1}
	{1+n^3}
\end{equation}
\vspace{2in}
\begin{equation}
	\sum
	\frac
	{n^3}
	{n^4-1}
\end{equation}
\vspace{2in}
\begin{equation}
	\sum
	\frac
	{\sin^2n}
	{n^2}
\end{equation}
\pagebreak
\section{Ratio Test}
\begin{equation}
	\sum
	\frac
	{n!}
	{2^n}
\end{equation}
\vspace{3in}
\begin{equation}
	\sum
	\frac
	{3^n}
	{n!}
\end{equation}
\pagebreak
\section{Power Series}
\begin{equation}
	f(x)=
	\frac
	{x^2}
	{(1+x^3)^2}
\end{equation}
\vspace{3.5in}
\begin{equation}
	f(x)=
	\frac
	{1}
	{1+4x^3}
\end{equation}
\vspace{3.5in}
\begin{equation}
	f(x)=
	\ln(1+x^2)
\end{equation}
\pagebreak
\end{document}
